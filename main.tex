\documentclass{article}

\usepackage[margin=1in]{geometry}
\usepackage[utf8]{inputenc}
\usepackage[english]{babel}
\usepackage[nottoc]{tocbibind}
\usepackage{amsmath}
\usepackage{amssymb}
\usepackage[hidelinks]{hyperref}
\usepackage[dvipsnames,table]{xcolor}
\usepackage{courier}

\definecolorset{RGB}{}{}{dark-grey,50,48,46;link-blue,0,0,153}

\hypersetup{%
  linktoc=all,%
  linkcolor=link-blue,%
  urlcolor=link-blue,%
  colorlinks=true,%
}
% Front matter

\title{Information Theory -- Homework 2}
\author{Zohreh Sadeghi\thanks{zsadeghi@uw.edu}}

% Custom commands

\newcommand{\shellprompt}[1]{{%
\vspace{0.3in}%
\hangindent=0.5in%
\small{\texttt{\$ #1}}%
\vspace{0.3in}%
}}

\begin{document}

\maketitle
\tableofcontents
\clearpage

\section{Parity Error of a Generator Matrix}

This conceptually makes sense, since the total error of a generator matrix using
its own parity matrix has to be zero.

However, we can mathematically prove this as well. For a given code $x$ generated
by generator matrix $G$, we have $x=xG$. Since $x$ is a valid code, we can use the
identity that its transposed form multiplied by the parity matrix will yield a zero
matrix (i.e. $Hx^T=\textbf{0}$).

Starting from here, we can write:

\begin{align*}
  &Hx^T=\textbf{0}\\
  &\Rightarrow H(xG)^T=\textbf{0}
\end{align*}

Using the lemma $(A.B)^T=B^T.A^T$, we can continue:

\begin{align*}
  &H(xG)^T=\textbf{0} \\
  &\Rightarrow HG^Tx^T=\textbf{0}
\end{align*}

Since $x$ is nonzero, we can write

\begin{align*}
  &HG^Tx^T=\textbf{0} \\
  &\Rightarrow HG^T=\textbf{0} \\
  &\Rightarrow (HG^T)^T=(\textbf{0})^T \\
  &\Rightarrow G.H^T=\textbf{0} \\
\end{align*}

\begin{flushright}
  $\blacksquare$
\end{flushright}

\section{Single-error-correcting Hamming Codes}

\paragraph{a.} For the first five non-trivial Hamming codes
(i.e. $r \in \{3, 4, 5, 6, 7\}$) we have:

\begin{itemize}
  \item \textbf{$r=3$} : $m=3n=2^3-1=7$; $k=2^3-1-3=4$. We have $[7,4]$.
  \item \textbf{$r=4$} : $m=4n=2^4-1=15$; $k=2^4-1-4=11$. We have $[15,11]$.
  \item \textbf{$r=5$} : $m=5n=2^5-1=31$; $k=2^5-1-5=26$. We have $[31,26]$.
  \item \textbf{$r=6$} : $m=6n=2^6-1=63$; $k=2^6-1-6=57$. We have $[63,57]$.
  \item \textbf{$r=7$} : $m=7n=2^7-1=127$; $k=2^7-1-7=120$. We have $[127,120]$.
\end{itemize}

\paragraph{b.} To calculate the rate, we calculate the fraction $\frac{k}{m}$
using the values we deduced for $m$ and $k$:

\begin{itemize}
  \item $r=3 \Rightarrow \frac{k}{m}=\frac{4}{7}=0.571428571$
  \item $r=4 \Rightarrow \frac{k}{m}=\frac{11}{15}=0.733333333$
  \item $r=5 \Rightarrow \frac{k}{m}=\frac{26}{31}=0.838709677$
  \item $r=6 \Rightarrow \frac{k}{m}=\frac{57}{63}=0.904761905$
  \item $r=7 \Rightarrow \frac{k}{m}=\frac{120}{127}=0.94488189$
\end{itemize}

\paragraph{c.} Given channel error probability $q$, we can derive the probability
$p_e$ for decoding errors in each case as such:

For code $C \subset \{0,1\}^n$, the probability with which we will get the \textit{correct}
codeword is:

\begin{align*}
  P_{\overline{e}}=(1-q)^m+mq(1-q)^{m-1}
\end{align*}

Therefore, the probability of decoding errors, will be the complement of this value,
yielding:

\begin{align*}
  P_e=1-P_{\overline{e}}=1-(1-q)^m-mq(1-q)^{m-1}
\end{align*}

Since $m=2^r-1$ we always have that $m\stackrel{2}{=}1$, therefore, this can be
rewritten as

\begin{align*}
  P_e&=1+(q-1)^m-mq(q-1)^{m-1} \\
     &=1+(q-1)(q-1)^{m-1}-mq(q-1)^{m-1}\\
     &=1+\Big(q-1-mq\Big)(q-1)^{m-1}\\
     &=1+\Big((1-m)q-1\Big)(q-1)^{m-1}\\
     &=1-\Big(1-(1-m)q\Big)(1-q)^{m-1}\\
     &=1-\Big(1+(m-1)q\Big)(1-q)^{m-1}
\end{align*}

Since $\Big(1+(m-1)q\Big)(1-q)^{m-1}$ is always nonnegative, this expression holds
so long as this value is not greater than $1$.

\section{Systematic Codes}

\paragraph{a.} For generator matrix

\begin{align*}
  G=\begin{bmatrix}
    1 & 0 & 1 & 0 & 1 & 1 \\
    0 & 1 & 1 & 1 & 0 & 1 \\
    0 & 1 & 1 & 0 & 1 & 0
  \end{bmatrix}
\end{align*}

To transform the matrix into systematic form $G'=[I_3|P]$ we will have:

\begin{align*}
  G_1=\begin{bmatrix}
    1 & 0 & 1 & 0 & 1 & 1 \\
    0 & 1 & 1 & 1 & 0 & 1 \\
    0 & 1 & 1 & 0 & 1 & 0
  \end{bmatrix}
\end{align*}

Starting with $(0,0)$:

\begin{align*}
  \Rightarrow G_2=\begin{bmatrix}
    1 & 0 & 1 & 0 & 1 & 1\\
    0 & 1 & 1 & 1 & 0 & 1\\
    0 & 1 & 1 & 0 & 1 & 0
  \end{bmatrix}
\end{align*}

Starting with $(1,1)$:

\begin{align*}
  \Rightarrow G_3=\begin{bmatrix}
    1 & 0 & 1 & 0 & 1 & 1\\
    0 & 1 & 1 & 1 & 0 & 1\\
    0 & 0 & 0 & 1 & 1 & 1
  \end{bmatrix}
\end{align*}

Starting with $(2,2)$:

\begin{align*}
  \Rightarrow G_4=\begin{bmatrix}
    1 & 0 & 0 & 1 & 1 & 1\\
    0 & 1 & 1 & 1 & 0 & 1\\
    0 & 0 & 1 & 0 & 1 & 1
  \end{bmatrix}
\end{align*}

Pivoting on $(2,2)$:

\begin{align*}
  \Rightarrow G_5=\begin{bmatrix}
    1 & 0 & 0 & 1 & 1 & 1\\
    0 & 1 & 0 & 1 & 1 & 0\\
    0 & 0 & 1 & 0 & 1 & 1
  \end{bmatrix}
\end{align*}

Finally, we have matrix $G'$:

\begin{align*}
  \Rightarrow G'=\begin{bmatrix}
    1 & 0 & 0 & 1 & 1 & 1\\
    0 & 1 & 0 & 1 & 1 & 0\\
    0 & 0 & 1 & 0 & 1 & 1
  \end{bmatrix}
\end{align*}

Given the above matrix, the parity check matrix of the form $H'=[-P^T|I_3]$
will be:

\begin{align*}
  H'=\begin{bmatrix}
    1 & 1 & 0 & 1 & 0 & 0 \\
    1 & 1 & 1 & 0 & 1 & 0 \\
    1 & 0 & 1 & 0 & 0 & 1
  \end{bmatrix}
\end{align*}

The code generated by this matrix will be:

\begin{align*}
  C&=\{x|H'x^T=0\}\\
   &=\{000000, 001000, 010000, 011000\}
\end{align*}

\paragraph{b.} Given the above codeset $C$, the standard array can be formed
using the coset leaders (i.e. $\{0,1\}^6-C$) ordered according to their weight.

Only the first 15 cosets are chosen for this example, and not all are included
since there are $64-4=60$ ($64$ being the total number of words within $\{0,1\}^6$
and $4$ being the number of valid codewords from $C$) of them.

\begin{center}
  \begin{tabular}{c|c c c}
    000000 & 001000 & 010000 & 011000 \\
    \hline
    000001 & 001001 & 010001 & 011001 \\
    000010 & 001010 & 010010 & 011010 \\
    000100 & 001100 & 010100 & 011100 \\
    001000 & 010000 & 011000 & 100000 \\
    100000 & 101000 & 110000 & 111000 \\
    000011 & 001011 & 010011 & 011011 \\
    000101 & 001101 & 010101 & 011101 \\
    001001 & 010001 & 011001 & 100001 \\
    010001 & 011001 & 100001 & 101001 \\
    100001 & 101001 & 110001 & 111001 \\
    000110 & 001110 & 010110 & 011110 \\
    001010 & 010010 & 011010 & 100010 \\
    010010 & 011010 & 100010 & 101010 \\
    100010 & 101010 & 110010 & 111010 \\
    100100 & 101100 & 110100 & 111100 \\
  \end{tabular}
\end{center}

\paragraph{c.} Given that the weight of a codeword is the number of its nonzero
bits, we have the following weight distribution in $C$ as can be seen above:

\begin{itemize}
  \item $|\{c|w(c)=\textbf{0} \land c \in C\}| = \textbf{1}$
  \item $|\{c|w(c)=\textbf{1} \land c \in C\}| = \textbf{2}$
  \item $|\{c|w(c)=\textbf{2} \land c \in C\}| = \textbf{1}$
  \item $|\{c|w(c)=\textbf{3} \land c \in C\}| = \textbf{0}$
  \item $|\{c|w(c)=\textbf{4} \land c \in C\}| = \textbf{0}$
  \item $|\{c|w(c)=\textbf{5} \land c \in C\}| = \textbf{0}$
  \item $|\{c|w(c)=\textbf{6} \land c \in C\}| = \textbf{0}$
\end{itemize}

\paragraph{d.} Given a 6-bit code block, here is the interpretation of the block:

\begin{center}
  \begin{tabular}{|c|c|c|c|c|c|}
    \hline
    \cellcolor{YellowGreen} $d_6$ & \cellcolor{YellowGreen} $d_5$ & \cellcolor{GreenYellow} $p_4$ & \cellcolor{YellowGreen} $d_3$ & \cellcolor{GreenYellow} $p_2$ & \cellcolor{GreenYellow} $p_1$ \\
    \hline
  \end{tabular}
\end{center}

where $d_i$ is a data bit and $p_i$ is a parity bit.

Considering

\begin{align*}
  & d_3 = 1 \\
  & d_5 = 0 \\
  & d_6 = 1 \\
  & p_1 = d_3 + d_5 \\
  & p_2 = d_3 + d_6 \\
  & p_4 = d_5 + d_6 \\
\end{align*}

with even parity, we will have the payload of the message as:

\begin{center}
  \begin{tabular}{|c|c|c|c|c|c|}
    \hline
    \cellcolor{YellowGreen} $1$ & \cellcolor{YellowGreen} $0$ & \cellcolor{GreenYellow} $1$ & \cellcolor{YellowGreen} $1$ & \cellcolor{GreenYellow} $0$ & \cellcolor{GreenYellow} $1$ \\
    \hline
  \end{tabular}
\end{center}

When receiving code $111001$, we can see that parity bits will add up to zero:

\begin{align*}
  (1 + 1 + 0)_2=0
\end{align*}

which means that we can read the message from the $d_i$ bits as $110$.

\section{Code Project}

The answer to this question has been implemented in Java and uses Apache Maven
as the build tool.

You have two options to run the code.

\subsection{Building from Source}

On a macOS machine you can download \href{http://brew.sh}{Homebrew} and then in a
Terminal window run:

\shellprompt{brew install maven}

to download and install Apache Maven.

The code can be found in \url{https://github.com/zsadeghi/info-theory-final}. You can
either use the web UI to download the contents of the repository as a ZIP file or you
can clone the repository using the command line:

\shellprompt{git clone https://github.com/zsadeghi/info-theory-final}

After which the relevant directories will be copied into your machine. You can then
go to the directory containing the code:

\shellprompt{cd info-theory-final/encdec/main}

This folder contains a \texttt{pom.xml} file which can be used to build the project:

\shellprompt{mvn clean install}

This command creates a directory called \texttt{target} under which you can find the
executable JAR file of the project:

\shellprompt{java -jar target/encdec.jar}

This JAR file optionally takes 2 arguments for \texttt{n} and \texttt{k}.

\subsection{Download the JAR File}

You can download the prebuilt JAR file of the project from \href{https://github.com/zsadeghi/info-theory-final/releases/tag/release}{the releases section}
of the Github repository for this project.

\end{document}
